\documentclass[12pt]{article}
\usepackage{}
%%%%%%%%%%%%%%%%%%%%%%%%
% PREAMBLE
\usepackage{times}
\usepackage{setspace}
\linespread{1.25}
\usepackage[letterpaper, margin=1in]{geometry}
\addtolength{\topmargin}{-0.5in}

\begin{document}
\title{Annotated Bibliography: Predicting the Ending Notes of Bach Chorales}
\author{Nicholas Czaban}
\maketitle

\noindent \hangindent=0.7cm Boyd, Malcolm. Bach, Oxford University Press, 2001. ProQuest Ebook Central, https://0-ebookcentral-proquest-com.dewey2.library.denison.edu/lib/wooster/detail.action?docID=281186.\\

This biography by Malcolm Boyd follows through Bach's life and looks at the various influences on the composer's music. Although the author has a very thorough section on the chorales Bach composed, his focus is more on the keyboard accompinament, rather than the vocalists themselves. Furthermore, there is not a great deal of in-depth analysis of the chorales in terms of their structure or progressions, although there are a few details mentioned here and there. However, the broader focus of the book will be able to back up or refute any inferences to be gained from my analysis of the chorale data.\\

\noindent \hangindent=0.7cm De Felice, Clelia, et al. "Splicing music composition." Information Sciences, vol. 385 - 386, 2017, pp. 196-212. OhioLINK Electronic Journal Center, doi:10.1016/J.INS.2017.01.004.\\

In this article, the authors suggest a unique application of splicing systems, a model originally concieved to study DNA recombination. The authors apply the methodology of this splicing system to train a computer to compose new music. The authors also calculated the overall probability of a particular chord progression being used. This source will be an excellent reference for cross-verification. Additionally, the computer-generated pieces could be used in an extension of the project once the model is complete.\\

\noindent \hangindent=0.7cm Dua, D. and Karra Taniskidou, E. (2017). UCI Machine Learning Repository [http://archive.ics.uci.edu/ml]. Irvine, CA: University of California, School of Information and Computer Science.\\

This data set contains the melody lines from 100 Bach chorales, containing information about each note in the piece. Every note of each chorale contains 6 fields of information: start-time of note, pitch of note, duration of note, key signature of note, time signature of note, and whether or not the note is under a fermata.\\

\noindent \hangindent=0.7cm Miranda, Eduardo R. Composing music with Computers. New York, NY: Routledge, 2014. Safari Books Online, http://0-proquest.safaribooksonline.com.dewey2.library.denison.edu\\/book/audio/9780240515670.\\

This comprehensive text on autonomous composition spends much of its time on the necessary math and computer science knowledge needed to create an automated composer. It covers two major techniques: nueral computation and iterative algorithms. Both of these methods could provide excellent starting points for developing my own model.\\

\noindent \hangindent=0.7cm Ultan, Lloyd. Music Theory : Problems and Practices in the Middle Ages and Renaissance, University of Minnesota Press, 1973. ProQuest Ebook Central, https://0-ebookcentral-proquest-com.dewey2.library.denison.edu/lib/wooster/detail.action?docID=316566.\\

This text provides excellent research on the development of music during the Middle Ages and Renaissance, leading up to and including the early portion of the common practice period. Each chapter focuses on the chronological development of various conventions, with sections on both the musical devices and the larger format of pieces.
\end{document}